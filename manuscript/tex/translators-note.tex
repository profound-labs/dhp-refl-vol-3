
Nesta tradução para português dos versos do Dhammapada aqui apresentados por Ajahn Munindo, pesquisaram-se outras publicações com o objectivo de melhor expor na língua portuguesa, os relatos do Buddha. Para tal, foi realizado um estudo sobre o contexto histórico da época e das diversas histórias que deram origem aos diferentes versos, assim bem como dos vários termos Pāli, que por virem de tão diferente contexto cultural, requerem por vezes uma explanação elaborada, visto não encontrarem parceria nas palavras ocidentais. Isto reflecte toda uma outra forma de ser, de estar e de pensar, presente na cultura oriental. 

No desenrolar desta pesquisa foram consultadas várias edições do Dhammapada, de forma a oferecer ao leitor uma tradução o mais precisa e coerente possível. Peço desde já desculpa por qualquer lapso ou imprecisão que possa ter ocorrido e fica aqui o meu agradecimento a Bhikkhu Thitavijjo e a Bhikkhu Vinita pela sua preciosa ajuda no estudo e aprofundamento do Pāli, e a Maria Ferreira da Silva e Helena Gallis pela sua preciosa ajuda no trabalho de revisão. 

Que este pequeno livro traga luz e alento a todos aqueles que o folhearem. Bem-hajam.

\bigskip
{\par\raggedleft
Appamādo Bhikkhu\\
2011
\par}
