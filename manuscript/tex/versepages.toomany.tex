
%% == 1 ==

\begin{dhpVerse}{2}
\label{dhp-2}
All states of being are determined by mind.\\
It is mind that leads the way.\\
As surely as our shadow never leaves us,\\
so well-being will follow\\
when we speak or act with a pure state of mind.
\end{dhpVerse}

\begin{dhpRefl}
  Coming at the very beginning of the \emph{Dhammapada}, this verse serves as a
  most powerful reminder of where our priorities need to lie. Almost constant
  sensory impingement means we easily become distracted and lose perspective, so
  we may forget that we are responsible for the way we view the world. But with
  skilful reflection we can remember, and ignite that right kind of effort which
  means we can fall back into awareness and ‘see’ things in a new way - a way
  which means that even if conditions are not agreeable, they are still just
  conditions. Changing conditions. Programmed conditions. We can remember not to
  believe too much in the way things appear. We can remember the heart’s warmth
  and the strength found in kindness, patience and clear-seeing. The trick is
  how to remember.
\end{dhpRefl}

%% == 2 ==

\begin{dhpVerse}{21}
\label{dhp-21}
Appreciative awareness leads to life;\\
heedless avoidance is the path to death.\\
Those who are truly aware are fully alive,\\
while those who are heedless\\
are as if already dead.
\end{dhpVerse}

\begin{dhpRefl}
  We all know the Buddha praised the cultivation of awareness. But how do we
  know the right thing to be aware of in any given moment? The objects are so
  varied: sights, sounds, smells, tastes, sensations and mental impressions. One
  exercise in awareness could be simply attending to the changing, unstable
  nature of all things, until we start to see them as unreliable, not really
  worth clinging to. This is to be aware of the characteristic of the ‘contents’
  of experience. What happens if we direct awareness towards the ‘context’ of
  experience? Is the characteristic of the context in which all objects of
  attention manifest the same?
\end{dhpRefl}

%% == 3 ==

\begin{dhpVerse}{166}
\label{dhp-166}
Knowing the Way for oneself,\\
walk it thoroughly.\\
Do not allow the needs of others,\\
however demanding,\\
to bring about distraction.
\end{dhpVerse}

\begin{dhpRefl}
  At first glance this verse could sound as if we are being taught not to care
  about others, but the Buddha is really pointing to where the right emphasis in
  our practice lies. If there is an oxygen shortage in an aeroplane, the captain
  tells the passengers to put on their own masks first, before attempting to
  help others. This is not to stop parents helping their children, but so that
  they can help them. Under stress we can lose perspective and act in ways that
  make things worse. The Buddha saw how easily distracted and confused
  unenlightened people could become, and when he had announced he was dying, he
  said that those who were serious about honouring him would not concern
  themselves with bringing him flowers. Instead they would intensify their focus
  on practising what he had taught.
\end{dhpRefl}

%% == 4 ==

\begin{dhpVerse}{275}
\label{dhp-275}
If you walk the path,\\
you will arrive at the end of suffering.\\
Having beheld this myself,\\
I proclaim the Way which removes all thorns.
\end{dhpVerse}

\begin{dhpRefl}
  Life hurts. The most natural thing is to seek a way of being in this world
  that is free from hurt. Those who have walked this way before us speak about
  the indescribable sense of relief on reaching the land of freedom, but they
  also point out that to reach it requires skilful effort. The Buddha delivered
  this verse to a group of monks as they were chatting about a trip they had
  been on together. He directed the discussion away from talk concerning the
  roads and rivers they had crossed, towards consideration of their inner
  terrain. The Teacher advises us to use our limited time and energy in a way
  that takes us in the direction we most want to go.
\end{dhpRefl}

%% == 5 ==

\begin{dhpVerse}{22}
\label{dhp-22}
The wise, being fully alive,\\
rejoice in appreciative awareness\\
and abide delighting in this capacity.
\end{dhpVerse}

\begin{dhpRefl}
  How capable are we of truly appreciating that which is in front of us? And can
  we appreciate that which appreciates? We have the capacity to know, but do we
  know accurately? For the Buddha and the great disciples, awareness was
  unobstructed. In and of itself, awareness was a source of joy. They knew the
  range of human experiences, yet were never lost in them. They simply knew, and
  their actions of body, speech and mind were an expression of this clear
  seeing. Our discriminating faculties mean we are able to manipulate conditions
  and create material comfort and safety. But are we able to stop manipulating
  them and appreciate them in their true light?
\end{dhpRefl}

%% == 6 ==

\begin{dhpVerse}{33}
\label{dhp-33}
Just as a fletcher shapes an arrow,\\
so the wise develop the mind;\\
so excitable, uncertain\\
and difficult to control.
\end{dhpVerse}

\begin{dhpRefl}
  Bringing body and mind in line with that which is true calls for a special
  kind of skill. Just when we thought we had our spiritual practice in order, we
  trip and fall again. But it doesn’t matter if we stumble from time to time.
  Learning to walk is like that, falling over is part of it. The skill worth
  developing is the agility which finds us readily picking ourselves up and
  beginning again; without looking back.
\end{dhpRefl}

%% == 7 ==

\begin{dhpVerse}{193}
\label{dhp-193}
It is hard to find\\
a being of great wisdom;\\
rare are the places in which they are born.\\
Those who surround them when they appear\\
know good fortune indeed. 
\end{dhpVerse}

\begin{dhpRefl}
  With appreciation for the good fortune of having ready access to the Buddha's
  teachings, we apply those teachings carefully so that they truly serve our
  liberation, not our being trapped. An initial glance at this Dhammapada verse
  might cause us to feel hard done by, since we are probably not living next
  door to enlightened beings. A more considered reading, however, leads us to
  recognize inner wisdom, here and now, and witness the profound effect that one
  moment of insight can have on the rest of our life. As we let go of a fixed
  view, our being is transformed accordingly. Its total transformation does not
  depend on finding someone else to enlighten us; it depends on our maintaining
  the effort to live life in full awareness, one moment at a time, allowing the
  gratitude which naturally arises as our understanding increases to nourish our
  heart. The letter of this verse may well refer to the good fortune of living
  near those who are wise, but the spirit encourages us to look inwards and
  value selfless wisdom wherever it manifests: to absorb it, cultivate it, and
  enjoy~it.
\end{dhpRefl}

%% == 8 ==

\begin{dhpVerse}{418}
\label{dhp-418}
Those who cease\\
to set up like against dislike,\\
who are cooled,\\
who are not swayed\\
by worldly conditions --\\
these I call great beings.
\end{dhpVerse}

\begin{dhpRefl}
  Liking and disliking can happen so quickly that we may feel we have no control
  over them. Somebody says something pleasant and we like them. Another person
  says something hurtful and we dislike them. It might be true that we can't
  stop liking and disliking arising, but if we slow down a little we might
  notice that in fact we do have a choice; whether or not to follow the liking
  or disliking, whether or not to make a ‘me’ out of them. When awareness is
  well established, liking and disliking can be seen as movement taking place in
  a larger reality. What is that reality?
\end{dhpRefl}

%% == 9 ==

\begin{dhpVerse}{164}
\label{dhp-164}
Like the bamboo\\
which destroys itself as it bears fruit,\\
so fools harm themselves\\
by holding to wrong views\\
and deriding those worthy ones\\
who live in harmony with the Way.
\end{dhpVerse}

\begin{dhpRefl}
  Believing all natural things are lovely is just a romantic notion. Nature can
  be cruel and deadly. The nature of our untrained passions can cause us to
  grasp at thoughts and feelings which only drag us down into increased
  suffering. Our untrained minds may cause us to miss the potential for freedom
  that could be ours if we made the effort to cultivate all-round right
  mindfulness, as the teachings encourage us to do: sitting, standing, walking
  and lying down; mindfulness 24/7. A mind thoroughly prepared for a life of
  awareness is constantly alert to the tendency to grasp at views and the risk
  of becoming possessed by self-importance. Until we are fully released from
  ignorance, the seeds of delusion remain within us, with their potential for
  causing harm to ourselves and others. A well-trained mind sees where virtue
  lies, even when our dearest opinions are contradicted and our personal
  preferences frustrated.
\end{dhpRefl}

%% == 10 ==

\begin{dhpVerse}{113}
\label{dhp-113}
A single day lived in awareness\\
of the transient nature of life\\
is of greater value than a hundred years\\
lived unaware of birth and death.
\end{dhpVerse}

\begin{dhpRefl}
  Change is always with us, but we are not always aware of it. Sometimes it is
  glaringly obvious, while at other times it is hard to discern. Indeed, the
  very perception of 'us' is constantly changing. How can we accord with this
  truth of impermanence without creating unnecessary suffering? The Buddha's
  advice is to know it fully, constantly, not to forget it and not to assume
  that anything is going to last forever. Such wise reflection is an essential
  aspect of the Way. Slowly but surely we establish pathways in our thinking
  that accord with Truth. Change is not wrong unless we label it as such. In
  fact, it is just so. It might be inconvenient from the perspective of
  preference; it can certainly hurt; but it will only produce suffering if we
  deny it. The Buddha also pointed out that realizing the inherently changing
  nature of all things leads to ultimate bliss.
\end{dhpRefl}

%% == 11 ==

\begin{dhpVerse}{421}
\label{dhp-421}
Anyone who lives freed\\
from habits of clinging\\
to past, present or future,\\
possessing nothing,\\
is a great being.
\end{dhpVerse}

\begin{dhpRefl}
  The momentum of our inner story-telling can be intimidating. Or perhaps we
  have to endure inner movies over and over again, rearranging fragments of
  memories with ourselves in the leading role. Probably our personal history
  does not really warrant such attention, and surely we would stop the inner
  noise if we could. So what feeds this momentum? Judging, taking sides,
  accepting and rejecting. Once again we are reminded of the need for
  here-and-now, whole body-mind, judgement-free awareness, with particular
  emphasis on the latter. How do we free awareness from the compulsion to take
  sides, to judge? We watch it. With a frame of reference established in the
  body, we gain a perspective on the habitual mental activity that we experience
  as judging, and we learn not to judge it, not to judge the judging mind. If we
  stop judging the inner activity, its momentum could cease.
\end{dhpRefl}

%% == 12 ==

\begin{dhpVerse}{385}
\label{dhp-385}
I say a being is great\\
who stands not on this shore\\
nor the other shore,\\
nor on any shore at all.\\
Such a being is free from all ties.
\end{dhpVerse}

\begin{dhpRefl}
  What type of person can be considered great? Where can excellence be found in
  our human society? This matters because what we are searching for is the
  quality or person we would emulate in pursuit of fulfilment. The Buddha says a
  great being holds to no fixed position, but is free from all attachments, not
  obstructed by clinging to material possessions or views. Such beings are not
  irresponsible in their actions; in fact, only they are fully responsible. One
  who is unattached can see clearly, feel accurately and respond truly.
  Everything is always changing, and holding to fixed views and attaching to
  material possessions just leads to stress. In confusion we strive to cling
  more, hoping that will help. What really helps is learning the right way to
  let go.
\end{dhpRefl}

%% == 13 ==

\begin{dhpVerse}{245}
\label{dhp-245}
Life is not easy for those\\
who have a sense of shame,\\
who are modest, pure-minded and detached,\\
morally upright and reflective.
\end{dhpVerse}

\begin{dhpRefl}
  If we find ourselves thinking, 'This is just too much, I can't let go of this
  one', we need to be extra careful. It is easy to let go of minor attachments,
  but the really serious ones are a different story. The Buddha knew about that
  different story, the one we tend to believe when faced with deep attachments.
  But the truth is still the truth, no matter how hard it feels; all suffering
  is rooted in attachment, which is why the Buddha gave us teachings like this
  verse. It really is hard to remain true when the forces of delusion pull at
  us. Whether it is the outer influences of sense objects or the inner currents
  of conditioning which tell us we are weak and unable, we resolutely seek the
  power to override those forces by turning to the Refuges. This is not just
  replacing a negative story with a positive one, but calling on reality to be
  our refuge.
\end{dhpRefl}

%% == 14 ==

\begin{dhpVerse}{132}
\label{dhp-132}
To avoid bringing harm to living beings\\
who, like us, seek contentment,\\
is to bring happiness to ourselves.
\end{dhpVerse}

\begin{dhpRefl}
  Sometimes what we do leads to happiness. At other times what we don’t do leads
  to happiness. Here the Buddha is saying that we should be aware of how the
  effort not to harm living beings brings happiness, not just for those who
  therefore live free from fear and pain, but also for ourselves. In a world
  where our worth is often measured in terms of our productivity, we can become
  caught in feeling that we always have to do something to make things better.
  At times, though, it is restraint that matters most.
\end{dhpRefl}

%% == 15 ==

\begin{dhpVerse}{226}
\label{dhp-226}
All pollution is cleared\\
from the minds of those who are vigilant,\\
training themselves day and night,\\
and whose lives are fully intent upon liberation.
\end{dhpVerse}

\begin{dhpRefl}
  If we are confident about the path and direction in which we are travelling,
  we won’t spend time in aimless wandering. Hence the teachings encourage us to
  be vigilant. Energy and devotion to spiritual practice are not enough,
  however. Our lives need to be directed towards the right goal: experiencing
  for ourselves the state of perfect freedom from suffering. As we travel along
  the path, we can check to see whether our progress and direction are correct
  by observing the pollutions of mind. Greed, ill will, laziness, anxiety,
  hesitation: are they diminishing or are they increasing? We get to know the
  reality of the pollutions as and when they appear. We will learn about their
  disappearance, here and now, in the same way.
\end{dhpRefl}

%% == 16 ==

\begin{dhpVerse}{25}
\label{dhp-25}
By endeavour, vigilance,\\
restraint and self-control,\\
let the wise make islands of themselves\\
which no flood can overwhelm.
\end{dhpVerse}

\begin{dhpRefl}
  It may seem at times as if the Buddha contradicts himself. He teaches us to
  focus on developing loving-kindness towards all beings, but here he appears to
  be saying we should isolate ourselves. But all the Buddha’s teachings are
  ‘pointings’; they are not fixed positions. He was often asked to state where
  he stood in relation to a particular philosophical view. In all cases he tried
  to avoid giving the questioner any view to cling to. He didn’t stand on
  anything; he accorded with Dhamma as it manifested. The path he taught is
  \emph{Dhammavicaya}, investigation of reality. He said that he could only
  point the way; it is up to us to travel the way he showed.
\end{dhpRefl}

%% == 17 ==

\begin{dhpVerse}{29}
\label{dhp-29}
Aware among those who are heedless,\\
awake among those who sleep,\\
the wise go forward like strong young horses,\\
leaving the exhausted behind.
\end{dhpVerse}

\begin{dhpRefl}
  How can we stay true to our heart’s aspiration for awakening and yet remain
  sensitive to the needs of those around us? If we lose balance, our good
  intentions lead only to more suffering. The teachings tell us to take care
  that our effort in practice is informed by wisdom and not by preferences.
  Being sensitive to the needs of others is important, but not at the expense of
  the quality of our own awareness. If the distortion of selfishness dominates
  consciousness, our efforts will fail to benefit us. Firmly establishing this
  sense of priority, we build an inner foundation. Our response to criticism can
  be kindness and understanding rather than defensiveness and anger.
\end{dhpRefl}

%% == 18 ==

\begin{dhpVerse}{236}
\label{dhp-236}
Hasten to cultivate wisdom.\\
Make an island for yourself.\\
Freed from stain and defilement\\
you will enter noble being.
\end{dhpVerse}

\begin{dhpRefl}
  We arrive at the state of noble being when we are at one with who we truly
  are. Our work is to recognize when we become false, pretending to be something
  or somebody we are not. Stains and defilements appear when we believe in the
  stories the mind tells us. If we feel resentment, we simply need to see the
  feeling of resentment clearly. If we feel fear, we simply need to see that
  feeling clearly. We don't need to pretend. We 'make an island' for ourselves
  by establishing awareness as the foundation in our life. This clear-seeing
  awareness can tell the difference between the real and the false, leading to
  freedom.
\end{dhpRefl}

%% == 19 ==

\begin{dhpVerse}{65}
\label{dhp-65}
Like the tongue that can appreciate\\
the flavour of the soup\\
is one who can clearly discern the truth\\
after only a brief association with the wise.
\end{dhpVerse}

\begin{dhpRefl}
  The number of retreats we go on is not as important as our ability to discern
  truth. The amount of time we spend sitting in meditation does not matter as
  much as our ability to see clearly what is in front of us. If our awareness is
  here-and-now, whole body-mind, and judgement-free, we can learn from all
  aspects of our life. If we have the good fortune to encounter wisdom in any
  form, we will recognize it. It won't have to appear Buddhist, up to date or
  even overtly wise. The heart will simply know it and be gladdened.
\end{dhpRefl}

%% == 20 ==

\begin{dhpVerse}{198}
\label{dhp-198}
While in the midst\\
of those who are troubled,\\
to dwell free from troubling\\
is happiness indeed.
\end{dhpVerse}

\begin{dhpRefl}
  When those around us are struggling, we might feel that somehow it is not
  quite right to be happy. The Buddha tells us the opposite. An external display
  of excessive delight would be out of place, but maintaining inner joy is
  perfectly suitable. In fact, to do so might be the very best thing we could
  offer to a troubled situation.
\end{dhpRefl}

%% == 21 ==

\begin{dhpVerse}{399}
\label{dhp-399}
Strength of patience\\
is the might of noble beings;\\
they can be shackled,\\
endure verbal abuse and beatings\\
without resorting to anger.
\end{dhpVerse}

\begin{dhpRefl}
  The force of self-righteousness within us needs taming. The more clever we
  are, the more careful we need to be. The more eloquent our speech, the more
  restraint is required. It is only when we know we can say 'no' to ourselves,
  when we know we don't always have to be the winner, that we can appreciate the
  transforming power of patient endurance.
\end{dhpRefl}

%% == 22 ==

\begin{dhpVerse}{234}
\label{dhp-234}
Ably self-restrained\\
are the wise,\\
in action, in thought\\
and in speech.
\end{dhpVerse}

\begin{dhpRefl}
  Restraint with awareness brings increased energy. Restraint motivated by fear
  depletes energy. Deluded ego’s way of meeting life is to try to control
  everything. It believes that without control things will fall apart -- a
  totally enervating way of living. However, so long as ego still believes it is
  the main player in this drama, we will struggle painfully. Even though we may
  feel we have faith in the Triple Gem, we will regularly fall back on our old
  ways of manipulating body, speech and mind. We fail to trust that true
  principles can guide us. This is because the momentum of ‘my way’ is untamed.
  But it is good if our insight in practice has produced enough wholesome doubt
  to start to undermine this false conviction. Now the work is to carefully
  question the apparent validity of ‘my way’. We can trust in this kind of
  doubt. It can help free us from the hallucination of self-importance.
\end{dhpRefl}

%% == 23 ==

\begin{dhpVerse}{224}
\label{dhp-224}
These three ways\\
lead to radiant abiding:\\
asserting the truth,\\
not yielding to anger\\
and giving, even if you have only a little to share.
\end{dhpVerse}

\begin{dhpRefl}
  We are the creators of the world. Our actions of body, speech and mind give
  form to the space we inhabit. Investing in inner awareness liberates us from
  dependence on the material world. The outer conditions of our life come and
  go: sometimes they are agreeable and rewarding, at other times wearisome and
  disappointing. But we can always make the effort to speak truth. We can always
  wait before succumbing to anger. And no matter how much or how little we own,
  we can always give. Thus we already have the power to create a beautiful
  abiding.
\end{dhpRefl}

%% == 24 ==

\begin{dhpVerse}{229-230}
\label{dhp-229}\label{dhp-230}
Those who live impeccably,\\
who are discerning, intelligent and virtuous --\\
they are continually praised by the wise.\\
Who would cast blame on those\\
who in their being are like gold?\\
Even the gods praise them.
\end{dhpVerse}

\begin{dhpRefl}
  Who do we compare ourselves with? The mental habit of comparing ourselves with
  others is mostly an expression of inner confusion, leading to more
  unhappiness. For the radiance of our true being to shine freely, all
  compulsive comparing must cease, but while we are still suffering from this
  habit it doesn’t help if we merely hold images of those more popular, wealthy
  or beautiful in our mind. It is better to compare ourselves with those who
  live impeccably, those who are more awake. Some of the Buddha’s greatest
  disciples were neither popular nor beautiful, but were greatly admired by all
  who could see clearly.
\end{dhpRefl}

%% == 25 ==

\begin{dhpVerse}{319}
\label{dhp-319}
The clear seeing which knows\\
that which is flawed as flawed\\
and that which is pure as pure\\
can lead beings to transcend misery.
\end{dhpVerse}

\begin{dhpRefl}
  The Buddha’s realization gives us a vision of life lived free from misery.
  Even if surrounded by those caught in the vortices of greed, aversion and
  delusion, Awakened Ones remain in a state of vitality and awareness. However,
  the path of practice leading towards this state might require us to find our
  way through swamps of doubt and across oceans of craving and fear. As we
  travel this inner terrain, we are simply asked to see clearly what is right in
  front of us. If we feel we’re drowning in desire or consumed by anger, we
  practise, not so as to create stories about how life could be, but simply to
  know it for what it is. Grasping at desire or anger does not lead to freedom
  or accord with well-being. And through practice we also see how letting go of
  grasping leads to contentment.
\end{dhpRefl}

%% == 26 ==

\begin{dhpVerse}{365-366}
\label{dhp-365}\label{dhp-366}
Bemoaning your own lot or envying the gains of others\\
obstructs peace of mind.\\
But being contented even with modest gains,\\
pure in livelihood and energetic,\\
you will be held in high esteem. 
\end{dhpVerse}

\begin{dhpRefl}
  This simple truth easily escapes us. Sadly, we are too quick to admire and
  emulate those who are not particularly wise. Here a very wise Teacher is
  holding up a mirror and asking, ‘Do you see what you are doing? Can you
  understand why you are unhappy?’ He is not criticizing or condemning us, but
  neither is he letting us get away with our habits. Out of compassion, he urges
  us to see the consequences of our unawareness. At times it can appear that
  there is always something more we need to do, to gain, to get rid of. Even the
  spiritual life can seem like a tedious treadmill. Always believing in the way
  things seem, however, is not the way to peace. Contentment could take the
  place of self-pity if we stop heedlessly comparing ourselves with others.
\end{dhpRefl}

%% == 27 ==

\begin{dhpVerse}{271-272}
\label{dhp-271}\label{dhp-272}
Do not rest contented because you keep\\
all the rules and regulations,\\
or because you achieve great learning.\\
Do not feel satisfied because you attain meditative absorption,\\
or because you can dwell in the bliss of solitude.\\
Only when you arrive at the complete eradication\\
of all ignorance and conceit should you be content.
\end{dhpVerse}

\begin{dhpRefl}
  Reading or hearing such profound teaching might give rise to a sense of
  urgency in practice - or it might cause us to give up because we feel we can’t
  do it. How we engage with ideals determines whether we are strengthened or
  weakened by them; the ideals themselves are not responsible. It matters that
  our ideals should accord with Truth, but it also matters that we shouldn't
  mistake an image of the goal for the goal itself. The Buddha wanted us to aim
  high; as high as can be and then further, but he didn’t want us to grasp the
  ideal and ignore our lowliness. Like a compass, the image of the goal offers
  direction - but of course, we don’t keep our eyes constantly on the compass.
  So long as we are heading in the right direction, we practise with that which
  is directly in front of us.
\end{dhpRefl}

%% == 28 ==

\begin{dhpVerse}{39}
\label{dhp-39}
There is no fear if the heart\\
is uncontaminated by the passions\\
and the mind is free from ill-will.\\
Seeing beyond good and evil,\\
one is awake.
\end{dhpVerse}

\begin{dhpRefl}
  The idea of fearlessness is deeply appealing. However, rather than seeing it
  as a remote goal somewhere out there, could we consider it as the most natural
  state within? From that perspective the states of fear we regularly endure can
  be considered as unnatural, not who and what we are. Surely it is unawareness
  that allows greed and resentment to contaminate our hearts, giving rise to
  fear. Adding fuel to those fires by thinking, ‘It shouldn’t be this way’
  doesn’t help. What does help is to make the right kind of effort to develop
  awareness and trust in the Buddha's Awakening.
\end{dhpRefl}

%% == 29 ==

\begin{dhpVerse}{402}
\label{dhp-402}
One who has realized the freedom\\
of having laid aside the burden\\
of identification with the body-mind\\
I call a great being.
\end{dhpVerse}

\begin{dhpRefl}
  It is good fortune to have reliable friends, to be healthy and to have
  received a good education. It is certainly good fortune to have found
  spiritual teachings that show a way across the swamps of delusion. It would be
  great good fortune if we could fully surrender ourselves to these teachings
  and experience the falling away of all burdens. In the meantime, we don't have
  to consider those so-called burdens as something going wrong. They might
  appear to obstruct our progress on the path, but they can also help us. In
  reality they are the natural consequence of unawareness, manifesting in each
  moment when we cling to the body-mind. The Buddha lived with limitless
  awareness, free from all clinging. Surrendering to his teachings means we
  don’t turn away from apparent obstructions; we study them with interest, with
  willingness, until we can see them for the fantasy they truly are.
\end{dhpRefl}

%% == 30 ==

\begin{dhpVerse}{82}
\label{dhp-82}
On hearing true teachings,\\
the hearts of those who are receptive\\
become serene,\\
like a lake, deep, clear and still. 
\end{dhpVerse}

\begin{dhpRefl}
  It is said that immediately following his Enlightenment the Buddha was
  disinclined to teach. Perhaps, seeing the extent to which we insist on
  creating suffering for ourselves and each other, and how we further compound
  that suffering by blaming others for it, he thought there was no point.
  Thankfully, however, despite the evidence of our foolishness, he eventually
  responded by teaching Dhamma. When selfless wisdom sees suffering, selfless
  compassion is the response. When he observed beings lost in habits of liking
  and disliking, he did all he could to help them let go. Being free from liking
  and disliking, Awakened Ones have an unobstructed view of all experience. It
  is not that they don’t feel as we feel; they just don’t get lost in
  experience.
\end{dhpRefl}

%% == 31 ==

\begin{dhpVerse}{352}
\label{dhp-352}
A master is one who has let go\\
of all craving and clinging to the world;\\
who has seen the truth beyond forms,\\
yet is possessed of a profound knowledge of words.\\
Such a great being can be said to have finished the task.
\end{dhpVerse}

\begin{dhpRefl}
  Letting go is not something we do, it is something which happens when we see
  how what we do causes suffering. So long as we are caught in trying to let go,
  the 'me' which is trying creates imbalance. But not trying isn’t correct
  either. What can we do to fulfil the great task of finding freedom? What does
  it mean to make right effort? One aspect of right effort is examining the kind
  of effort we are already making. Is what we are doing a form of self-seeking,
  or does it come from a deeper, quieter place, a simple interest in what is
  true? We know we want to be free from suffering, but does our way of wanting
  that freedom actually help? Even wanting to be free can create obstructions if
  we cling to it. Our aspiration to see ‘the truth beyond forms’ can support
  right effort if we slow down, remember kindness and examine how we are
  receiving our present experience.
\end{dhpRefl}

%% == 32 ==

\begin{dhpVerse}{327}
\label{dhp-327}
As an elephant\\
resolutely drags itself from a swamp,\\
uplift yourself with the inspiration\\
of cultivated attention.
\end{dhpVerse}

\begin{dhpRefl}
  The energy of inspiration can be generated by wise reflection. With the right
  kind of effort insurmountable situations can be managed, the unendurable can
  be endured. Inspiration has the power to transform our lives and our world.
  When wise reflection shows us that heedfulness helps and heedlessness hinders;
  our hearts respond by inclining towards the helpful. Balanced awareness
  rightly reveals the extent of the task we have ahead of us, with our inner
  world obstructed by ignorance and our outer world fraught with injustice. But
  the important question is how we meet this task. What is needed is not more
  force, but careful consideration of cause and effect. If clear seeing and
  kindness were to motivate us, the swamp of heedless habits would appear less
  daunting. Cultivated attention shows us what works, and confidence naturally
  follows.
\end{dhpRefl}

%% == 33 ==

\begin{dhpVerse}{96}
\label{dhp-96}
Those who arrive at the state of perfect freedom\\
through right understanding\\
are unperturbed in body, speech or mind.\\
They remain unshaken by life's vicissitudes.
\end{dhpVerse}

\begin{dhpRefl}
  The very best way to accommodate uncertainty is through right understanding,
  or right view. To expect to be always at ease with uncertainty would be naive,
  but we shouldn’t assume that we must be defined by it. Life, change and all
  the rest of it might appear to be too much, but life itself is never too much;
  it is always ‘just so’. If it was really too much, the Buddha could never have
  realized freedom while still alive. The view we hold is what makes the
  difference. If we take ourselves too seriously the situation can seem
  intolerable; we become tense, limiting possibilities for insight and
  sensitivity. Instead we could try relaxing and imagining an unconditioned
  reality in which all the changing conditions simply appear to arise and cease.
  Wise letting go leads to an expanded awareness and a fresh perspective on what
  we were doing which made it seem we had a problem.
\end{dhpRefl}

%% == 34 ==

\begin{dhpVerse}{11-12}
\label{dhp-11}\label{dhp-12}
Mistaking the false for the real\\
and the real for the false,\\
one suffers a life of falsity.\\
But seeing the false as the false\\
and the real as the real,\\
one lives in the perfectly real.
\end{dhpVerse}

\begin{dhpRefl}
  In our heart of hearts we long for completion, for the ‘perfectly real’.
  Walking the Buddha’s path towards such perfection means we stop ignoring the
  consequences of our imperfections. As long as we live in an image of
  ourselves, resisting reality, we are unreal. At an early stage of practice,
  ideas about how we should be do have a function. But we need to let go of
  those ideas quite quickly and instead feel what it feels like to be ‘me’.
  Maybe it feels imperfect, unreal. That’s a good insight -- it is! Try asking
  who or what it is that is aware of such feelings.
\end{dhpRefl}

%% == 35 ==

\begin{dhpVerse}{182}
\label{dhp-182}
It is not easy to be born as a human being\\
and to live this mortal life.\\
It is not easy to have the opportunity to hear Dhamma\\
and rare for a Buddha to arise.
\end{dhpVerse}

\begin{dhpRefl}
  In terms of here and now, we are born as human beings whenever we have
  mindfulness and integrity. Because of our tendency to compromise Dhamma
  principles, this task becomes difficult; following preferences is much easier.
  However, merely to follow liking and disliking is not living as we could be
  living. Instead we could reflect on cause and effect, on what happened last
  time we allowed ourselves to become lost in experience. We are fortunate these
  days to have ready access to Dhamma, but the Dhamma is hard to hear because we
  have created obstructions by having followed our preferences for so long. When
  the Buddha was alive, some of his disciples became enlightened at once by
  simply listening as he taught the Dhamma. Why can’t we listen in the same way,
  get the message and drop the burden? If we did that, the 'Buddha' would appear
  here and now.
\end{dhpRefl}

%% == 36 ==

\begin{dhpVerse}{43}
\label{dhp-43}
Neither mother, father\\
nor any other relative\\
can give you the blessings generated\\
by your own well-directed heart.
\end{dhpVerse}

\begin{dhpRefl}
  If the heart is well-directed, we feel there is something we can fall back on
  when things get difficult. If we experience despair, disappointment,
  disillusionment, the heart doesn’t have to sink into hopelessness. Nor does it
  have to seek security in hope. The refuge we can fall back on is not any thing
  or state at all; it is a way. Having looked into the consequences of grasping
  for long enough, we now seek confidence in letting go of fixed positions. We
  still have opinions and preferences, but we are not so committed to finding
  security in them. Learning to let go is the way to generate blessings.
\end{dhpRefl}

%% == 37 ==

\begin{dhpVerse}{351}
\label{dhp-351}
Those who have reached the goal\\
no longer need to re-form;\\
they are free from fear and longing.\\
The thorns of existence have been removed.
\end{dhpVerse}

\begin{dhpRefl}
  If we were thoroughly free from fear, that would mean we could handle whatever
  intensity life might present to us. As it is, we tend to lose ourselves when
  love or hate enter our hearts. Instead of our abiding as vast awareness,
  capable of accommodating strong feelings, fear manifests as a collapse into
  obstructed awareness. By clinging to and identifying with desire we create
  fear and then cling to the fear. To let go of fear we must also let go of
  desire. This does not mean, however, that desire and fear disappear.
\end{dhpRefl}

%% == 38 ==

\begin{dhpVerse}{276}
\label{dhp-276}
The Awakened Ones can but point the way;\\
we must make the effort ourselves.\\
Those who reflect wisely and enter the path\\
are freed from the fetters of Mara.
\end{dhpVerse}

\begin{dhpRefl}
  'What effort should I make? Should I do something about this situation or
  simply watch my mind?' Such moments of not-knowing are precious. Uncertainty
  does not have to be seen as failing. Indeed, we might miss something important
  if we are in a hurry to push past it. The fact is that we don't know what to
  do and there is not necessarily any fault in that. If, however, we're
  completely caught in the momentum of wanting to escape suffering, we may miss
  learning from the truth of the situation as it is. With the confidence that
  comes from our commitment to precepts, we can afford to trust in being patient
  and aware of 'not knowing' and the uncomfortable feelings that come with it.
  We can feel the force of the momentum of wanting to get away from it, to
  'solve it', but stubbornly refuse to be pulled along. We can experiment with
  waiting until the feeling of being driven subsides, and then quietly listen to
  what intuition suggests we could do.
\end{dhpRefl}

%% == 39 ==

\begin{dhpVerse}{174}
\label{dhp-174}
If birds are trapped in a net\\
only a few will ever escape.\\
In this world of illusion\\
only a few see their way to liberation.
\end{dhpVerse}

\begin{dhpRefl}
  It is an advantage to have a variety of skilful means at hand as we go forward
  in practice. Remember that the deluded personality will employ powerfully
  persuasive arguments in its attempts to maintain self-importance. We need an
  extensive repertoire of skills to meet those arguments. If tranquillity is
  called for, we could put effort into honing our ability to focus more
  precisely. Or it might be that further study of the traditional teachings will
  quell the doubt that disturbs us. At another time we may need to talk things
  over with a trusted, respected friend before we learn the lesson of letting
  go. Or perhaps we should find someone who shows us how to adjust our posture
  so we don’t get a headache every time we sit, or who teaches us what
  transformative patience looks like. Then again, it might be a good long walk
  in the country, followed by a nice cup of tea, which helps us to drop whatever
  is bothering us. Agility!
\end{dhpRefl}

%% == 40 ==

\begin{dhpVerse}{276}
\label{dhp-276}
The Awakened Ones can but point the way;\\
we must make the effort ourselves.\\
Those who reflect wisely and enter the path\\
are freed from the fetters of Mara.
\end{dhpVerse}

\begin{dhpRefl}
  Force of habit defines our actions when we are not firmly established in
  mindfulness. These habit patterns are the fetters of Mara. To enter the path
  leading to freedom from such disappointing limitations requires wise
  reflection. If we wish, we are already free to choose to turn the light of
  attention inwards and enquire about truth. It is not easy to cultivate the
  discipline and skill required for insight and letting go to happen; we could
  also allow our attention to wander aimlessly. But this is not an imposition
  put upon us; it is a choice we willingly make when we see the consequences of
  heedlessness. What good fortune to have a chance to work on wisdom.
\end{dhpRefl}

%% == 41 ==

\begin{dhpVerse}{374}
\label{dhp-374}
When those who are wise dwell in contemplation\\
of the transient nature of the body-mind\\
and of all conditioned existence,\\
they experience joy and delight,\\
seeing through to the inherently secure.
\end{dhpVerse}

\begin{dhpRefl}
  All the Buddha's teachings point to that which is unchanging, undying,
  inherently satisfactory. To wake up from the dream we live in, we dwell in
  contemplation on the changing, dying, unsatisfactory nature of conditioned
  existence. In our dream world we believe that attaching to things as if they
  were ultimate will make us happy. With his ready access to the unconditioned
  reality, the Buddha knew that clinging to any aspect of conditioned reality
  was a direct route to disappointment. He didn't teach this so we would create
  a philosophy about how unreliable and regrettable everything is. He lived in
  this world as we do, but didn't suffer and go on about how sad it all is. Life
  can seem sad and regrettable so long as we are identified as the body-mind.
  The Buddha's identity was undefinable because he didn't cling to anything, and
  his happiness was unshakeable because it didn't depend on anything.
\end{dhpRefl}

%% == 42 ==

\begin{dhpVerse}{185}
\label{dhp-185}
Not insulting, not harming,\\
cultivating restraint,\\
with respect for the training,\\
modesty in eating, contentment\\
with one's dwelling place\\
and devotion to mindful intent;\\
this is the Teaching of the Buddha.
\end{dhpVerse}

\begin{dhpRefl}
  Modesty and contentment are not part of consumer culture, but they are part of
  Buddhist culture. It is true that we need enthusiasm and energy, commitment
  and concentration, if we wish to reach the goal of liberation. But too much of
  these ‘hard virtues’ and we create unnecessary obstructions for ourselves.
  When we are mindful, it is possible that we will know if we are out of balance
  and adjust accordingly. The ‘soft virtues’ of contentment, modesty, humility
  are less attractive to our spiritual egos, but they might be just what is
  needed for dropping the burden.
\end{dhpRefl}

%% == 43 ==

\begin{dhpVerse}{329}
\label{dhp-329}
\ldots\ if you cannot find a good companion
of integrity and wisdom,
then, like a king departing a conquered land
or a lone elephant wandering the forest,
walk alone.
\end{dhpVerse}

\begin{dhpRefl}
  Integrity is the foundation of our practice. Without it nothing develops. We
  might have eloquent speech, and perhaps our articles have been published in
  popular journals or online, but if integrity is lacking, practice hasn’t
  begun. Travelling the spiritual path alone is not a sign of failure; it may
  mean quite the opposite. If those around us are willing to compromise on
  impeccability, it is better for us to be alone.
\end{dhpRefl}

%% == 44 ==

\begin{dhpVerse}{38}
\label{dhp-38}
In one whose mind is unsteady,\\
whose heart is not prepared with true teachings,\\
whose faith is not matured,\\
the fullness of wisdom is not yet manifest.
\end{dhpVerse}

\begin{dhpRefl}
  This Dhammapada verse may describe many of us: brought up with \mbox{minimal}
  spiritual education, minds locked in thinking mode, and incapable of giving
  ourselves whole-heartedly to anything. Yet we trust that real wisdom exists
  and that we have a chance of realizing it. It is this ‘initial’ type of faith
  that got us started and brought us thus far. Now we must build on it. Once we
  have tasted the benefits of practice, faith is verified and manifests quite
  differently. It becomes a reliable source of energy. In the beginning we were
  motivated by an idea or an intuition. Now we are invited to trust in an
  awareness informed by experience. This feels like spending money earned by our
  own efforts rather than handed out by someone else.
\end{dhpRefl}

%% == 45 ==

\begin{dhpVerse}{36}
\label{dhp-36}
Though subtle, elusive and hard to see,\\
the protected and guarded mind\\
leads to ease of being.\\
One who is alert should tend and watch over this mind.
\end{dhpVerse}

\begin{dhpRefl}
  When we watch over this heart-mind we cultivate inner light. When light in our
  outer world is dim we are inclined to trip over things, or perhaps mistake a
  piece of rope for a snake and run away in a completely unnecessary panic.
  Similarly, a lack of inner illumination can cause us to react in ill-judged
  ways, destroying our heart’s natural sense of ease. It is because we don’t see
  states of mind clearly that we react and make things worse. For example,
  perhaps we still feel hurt by something which happened years ago, and have
  dwelt on bitterness ever since because we didn’t see the truth of our
  reaction. Forgiveness is not a synthetic virtue with which to paste over our
  bruises. Although the memory of what happened might remain, we always have the
  choice of whether or not to invest that memory with resentment. This practice
  is subtle and hard to see, but it is worth the effort.
\end{dhpRefl}

%% == 46 ==

\begin{dhpVerse}{248}
\label{dhp-248}
Whoever is intent on goodness\\
should know this:\\
a lack of self-restraint is disastrous.\\
Do not allow greed and misconduct\\
to prolong your misery.
\end{dhpVerse}

\begin{dhpRefl}
  The Buddha knows that life is not always easy. He knows that even the practice
  of observing the five precepts can be difficult. The story associated with
  this verse involves a group of five lay disciples who are each observing one
  or two of the five Buddhist precepts. They each insist that theirs is the most
  difficult to cultivate and therefore by implication the most worthy. Arguing
  among themselves they approach the Buddha, each wanting him to praise his own
  practice and support their view that the precepts being kept are the most
  important. Instead the Teacher admonishes them, saying that none of the
  precepts is easy to keep, nor are any of them unimportant, and that everyone
  should train themselves in all five.
\end{dhpRefl}

%% == 47 ==

\begin{dhpVerse}{18}
\label{dhp-18}
Here and hereafter those who\\
live their lives well abide in happiness.\\
They are filled with a natural appreciation\\
of virtue and they dwell in delight.
\end{dhpVerse}

\begin{dhpRefl}
  There is nothing special about Dhamma. Dhamma is what is natural. Because we
  live such unnatural lives, we can miss what is directly in front of us. When
  we are balanced and at ease our faculties function to serve well-being. When
  we are agitated and confused we lose perspective. It is then that we tend to
  forget we are in charge of our destiny. If we do good, goodness comes back to
  us. If we do evil, suffering comes back to us. This view is not naive, it is
  natural. But being truly natural is not easy. Ultimately we aim to go beyond
  good and evil, and dwell in unshakeable~peace.
\end{dhpRefl}

%% == 48 ==

\begin{dhpVerse}{393}
\label{dhp-393}
One should not be considered worthy of respect\\
because of birth or background or any outer sign.\\
It is purity and the realization of truth\\
that determine one's worth.
\end{dhpVerse}

\begin{dhpRefl}
  Liberated beings are never fooled by the way things appear to be. They know
  the difference between outer ‘form’ which the eye sees, and ‘actuality’ which
  the heart knows. They naturally respect and take delight in the inherent
  beauty of the ‘real’. Our awareness, however, is limited because of fixed
  views, and we must take care not to casually follow our mind’s conditioning.
  So long as we are unaware of truth, we are susceptible to being impressed by
  outer forms. Transient beauty, intense emotions, wealth; all these and more
  intimidate us into unhelpful desires, into wanting that which brings no
  lasting benefit. Whenever we offer respect towards the truth which is beyond
  intimidation, our affinity with that truth increases.
\end{dhpRefl}

%% == 49 ==

\begin{dhpVerse}{206}
\label{dhp-206}
It is always a pleasure\\
not to have to encounter fools.\\
It is always good to see noble beings,\\
and a delight to live with them.
\end{dhpVerse}

\begin{dhpRefl}
  The Buddha gave this short teaching referring to conditions in the outer
  world, and it is not difficult to agree. We can also contemplate the spirit of
  this message with reference to our inner world; our mind states. How does it
  feel when we generate foolish thoughts? What happens when we cease to indulge
  in them? What is the effect of witnessing our heart’s wholesome aspirations?
  Is it possible to dwell for extended periods in noble intentions?
\end{dhpRefl}

%% == 50 ==

\begin{dhpVerse}{212}
\label{dhp-212}
From endearment springs grief.\\
From endearment springs fear of loss.\\
Yet if one is free from endearment\\
there is no grief,\\
so how could there be fear?
\end{dhpVerse}

\begin{dhpRefl}
  One way of reading this text is that we are wrong to hold things dear: family,
  friends, memories. That initial interpretation blames the feelings themselves
  for our suffering. But the Buddha is not just talking about the feelings; he
  is pointing to how we might be free. Is it possible to feel endearment and be
  free at the same time? When he heard that his two chief disciples, Venerables
  Sariputta and Moggallana, had died, the Buddha commented that it was as if the
  sun and the moon had gone from the sky. That doesn’t sound like someone who
  didn't feel anything. Knowing the truth of feelings means that we no longer
  find our sense of identity in them, but letting go of them does not mean they
  disappear. In what are all these feelings arising and ceasing? That was the
  Buddha’s abiding; hence he could feel fully and freely, without suffering.
\end{dhpRefl}

%% == 51 ==

\begin{dhpVerse}{100}
\label{dhp-100}
A single word of truth\\
which calms the mind\\
is better to hear than a thousand\\
irrelevant words.
\end{dhpVerse}

\begin{dhpRefl}
  Listening to many hours of Dhamma talks might be helpful, but the Buddha says
  that even one word of Dhamma can be enough. What matters is whether that word
  truly touches our hearts. Does it ring true? Truth is what heals us, not mere
  words. Living in a world that is distracted by materialism, we often assume
  that more is better. Yet one small passport can get us comfortably through
  immigration and is worth more than a truckload of books. They are both paper -
  what is the difference? We already know we need to attend to quality, not just
  quantity. This Dhammapada verse encourages us to take our understanding
  deeper.
\end{dhpRefl}

%% == 52 ==

\begin{dhpVerse}{168}
\label{dhp-168}
Do not show false humility.\\
Stand firmly in relation to your goal.\\
Practice, well-observed,\\
leads to contentment\\
both now and in the future.
\end{dhpVerse}

\begin{dhpRefl}
  Contentment might arise because we don’t actually want anything. But it is
  also possible for our hearts to remain contented even when we do want things.
  For that to be true, however, we must want with wisdom. Wanting is a movement
  in mind which, if we are honest, we experience as a sort of discontent. As
  long as we feel that it is ‘me’ who is wanting, it is ‘me’ who feels
  discontented, and very often this same ‘me’ is driven to do something to
  dispel the uncomfortable feeling. When there is wisdom, though, we cease
  seeking identity in the activity of mind and sense the stillness behind
  movement. This is the wisdom that can free us from feeling driven.
\end{dhpRefl}

%% == 52+1 ==

\begin{dhpVerse}{183}
\label{dhp-183}
Refrain from doing evil,\\
cultivate that which is good;\\
purify the heart.\\
This is the Way of the Awakened Ones.
\end{dhpVerse}

\begin{dhpRefl}
  The first stage of cultivating the way is refraining from following all that
  is evil. It is about learning to say ‘no’ to ourselves when we need to. As a
  result, we discover later that we can say ‘yes’ without losing ourselves. If
  we don’t recognize our unwholesome impulses for what they are, we might think
  the bad stuff is only in other people. The second stage of cultivating the way
  is developing that which is good. Even if it is only a small moment of
  goodness, don’t dismiss it. The third stage is purifying our effort from the
  taint of ‘me’. Even when we have completely finished redecorating a room, the
  smell of paint fumes remains. Though our practice might be growing stronger,
  the sense of self-importance could be growing stronger too.
\end{dhpRefl}

%% == 52+2 ==

\begin{dhpVerse}{378}
\label{dhp-378}
I call them the peaceful ones\\
who are calm in body,\\
in speech and in mind,\\
and who are thoroughly purged\\
of all worldly obsessions.
\end{dhpVerse}

\begin{dhpRefl}
  When religions teach us to dismiss the material world they are not being
  helpful. The Buddha taught us to understand the material world, not dismiss
  it. Insofar as we understand it, we need not be obsessed by it. If we see it
  clearly, we can recognize both its potential for increasing the happiness of
  living beings and the risk of increasing suffering. Without this clear seeing,
  the pleasure that arises with gratification of desire, for example, looks like
  the path to peace and contentment, but actually it is not. Such gratification
  is merely a momentary relief from the irritation of wanting. The peace and
  contentment that we seek are the companions of clear seeing. All things of the
  world, the agreeable and the disagreeable, are changing. Truly seeing this is
  seeing changelessness.
\end{dhpRefl}

%% == 52+3 ==

\begin{dhpVerse}{205}
\label{dhp-205}
Tasting the flavour of solitude\\
and the nectar of peace,\\
those who drink the joy\\
that is the essence of reality\\
abide free from fear of evil.
\end{dhpVerse}

\begin{dhpRefl}
  Physicians advise us to nourish the body by healthy eating and taking regular
  exercise. The Buddha advises us to nourish the heart with Truth. If we allow
  ourselves to become too busy, we forget how refreshing it can be to spend time
  alone, to take time for ourselves, and a sense of discontent gradually
  increases until we believe we are inherently lacking. This perception might
  please the consumer culture, but it doesn’t give us inner strength. Spiritual
  practice sometimes involves daring to take less and trusting in our heart’s
  natural undefiled state.
\end{dhpRefl}

%% == 52+4 ==

\begin{dhpVerse}{197}
\label{dhp-197}
While in the midst\\
of those who hate,\\
to dwell free from hating\\
is happiness indeed.
\end{dhpVerse}

\begin{dhpRefl}
  Usually we equate happiness with getting what we want. Might there be other
  forms of happiness? For all of us there are times when we don’t get what we
  want, or we get what we definitely do not want. In this verse the Buddha is
  pointing to a quality of happiness which arises irrespective of whether or not
  we get what we want, a happiness which arises with wisdom. Wisdom knows that
  some conditions can be changed and some cannot. We can’t, for instance, stop
  others from feeling hatred, but we can make the effort not to be pulled into
  their anger. And despite what some may say, this is not quietism. It is taking
  responsibility for what is ours and maintaining equanimity towards that which
  is not.
\end{dhpRefl}

%% == 52+5 ==

\begin{dhpVerse}{391}
\label{dhp-391}
One who refrains from causing harm\\
by way of body, speech or mind\\
can be called a great being.
\end{dhpVerse}

\begin{dhpRefl}
  Greatness could be defined in terms of the power we have or the possessions we
  own, but in the mind of the Buddha it is better determined by how people
  conduct themselves. This is a very practical way of assessing how trustworthy
  people may be. Are they restrained in how they act and in what they say? Are
  they kind? We can’t tell what is happening inwardly, but we can observe the
  influence they have on the world around them.
\end{dhpRefl}

%% == 52+6 ==

\begin{dhpVerse}{52}
\label{dhp-52}
As a beautiful flower\\
with a delightful fragrance is pleasing,\\
so is wise and lovely speech\\
when matched with right action.
\end{dhpVerse}

\begin{dhpRefl}
  The physical eye sees beauty on the level of form. The inner eye sees beauty
  on the level of spirit. The two don’t always coincide. A beautiful-looking
  person might speak words that are devious and dishonest, leading to harm. A
  challenging and frustrating experience might soften our hearts and lead us to
  renounce hubris. Appreciating outer forms without becoming intoxicated takes
  skill.
\end{dhpRefl}

%% == 52+7 ==

\begin{dhpVerse}{7}
\label{dhp-7}
As a stormy wind can uproot a frail tree,\\
so one who holds heedlessly to pleasure,\\
indulges in food and is indolent\\
can be uprooted by Mara.
\end{dhpVerse}

\begin{dhpRefl}
  How can we know the right amount of things? Our senses and society often tell
  us that we need more. The global economy is based on conditioning us to
  believe we lack things. If our refuge is in an expanded awareness, freed from
  the compulsive habit of taking sides, we are in a position to contemplate the
  conditioning process. It is essential to recognize that we don’t have to be
  enslaved by our environment. The work of inner reflection can lead to a
  confidence independent of popular belief or cultural bias. We are allowed to
  experiment with not eating so much or having an opinion on everything. It’s
  fine to be quiet and cultivate contentment. Contentment doesn’t have to mean
  abdication. What matters is that when a storm strikes, does it blow us over?
  From where do we draw our strength?
\end{dhpRefl}

%% == 52+8 ==

\begin{dhpVerse}{164}
\label{dhp-164}
Like the bamboo which destroys itself as it bears fruit,\\
so fools harm themselves by holding to wrong views\\
and deriding those worthy ones\\
who live in harmony with the Way.
\end{dhpVerse}

\begin{dhpRefl}
  It is easy to criticize weakness in others. It is hard to recognize and remedy
  faults within ourselves. At one level it can even feel good to elevate
  ourselves as we put others down, but such feelings cannot be trusted.
  One-upmanship is unlikely to contribute to harmony. Of course, there are times
  when criticism is called for, but for it to be constructive we must have
  wholesome intention. To know our intention accurately probably means that we
  should slow down a little, wait and listen inwardly before we speak.
\end{dhpRefl}

%% == 52+9 ==

\begin{dhpVerse}{42}
\label{dhp-42}
More than a thief,\\
more than an enemy,\\
a misdirected heart\\
brings one to harm.
\end{dhpVerse}

\begin{dhpRefl}
  A misdirected heart leads us to harm when it obstructs access to the natural
  state of contentment. When awareness fails and we attach, the act of clinging
  feeds into confused thinking and divisive action. True, self-existing
  well-being arises effortlessly for those who are at one with what is; with
  Dhamma. A liberated being doesn't have to try to be contented or not be
  discontented. Having seen the suffering caused by clinging, all inclination to
  attach to fixed positions has gone. The unobstructed heart leads only to
  understanding and ease.
\end{dhpRefl}

%% == 52+10 ==

\begin{dhpVerse}{15}
\label{dhp-15}
When we see clearly\\
our own lack of virtue\\
we are filled with grief;\\
here and hereafter we grieve.
\end{dhpVerse}

\begin{dhpRefl}
  If we were to stub our toe and not feel pain we would be in trouble. Pain is a
  message saying, ‘Pay attention.' Similarly, if we were to act or speak cruelly
  without feeling remorse in our heart, we would be in trouble. How could we
  learn? Despite appearances, remorse is not something going wrong. It is there
  to protect us, a sort of immune system. We can listen to it, accept it, invite
  the pain in our hearts to teach us how not to follow heedlessness in future.
  Becoming lost in remorse will lead to guilt; we've missed the point. We don’t
  learn by taking delight in hating ourselves or others for making a mistake.
\end{dhpRefl}

%% == 52+11 ==

\begin{dhpVerse}{81}
\label{dhp-81}
As solid rock\\
is unshaken by the wind,\\
so are those with wisdom undisturbed,\\
whether by praise or blame. 
\end{dhpVerse}

\begin{dhpRefl}
  Sometimes we feel liked and appreciated, at other times we feel disliked and
  dismissed. Is there a way of staying steady while being praised and blamed,
  accepted and rejected? When the Buddha first delivered his teachings
  encouraging the cultivation of wisdom, he said that a wise perspective on life
  is what offers inner security and steadiness. For a long time he had been
  seeking a solution to the frustrations of life. Once he found freedom from all
  frustration and suffering, he formulated his teachings into what we now know
  as the Four Noble Truths: there is suffering; there is a cause; there is
  freedom; and there is a practice we can do to pursue that freedom. The image
  of a rock unshaken by the wind inspires us to cultivate this wisdom.
\end{dhpRefl}

%% == 52+12 ==

\begin{dhpVerse}{131}
\label{dhp-131}
To harm living beings who, like us, seek contentment,\\
is to bring harm to ourselves.
\end{dhpVerse}

\begin{dhpRefl}
  Self-interest can be used in our pursuit of right action. When faced with
  danger we easily feel threatened, our hearts become inflamed and wise
  discernment is obscured. However, instead of losing ourselves in defensive
  reaction, right training can help us remember that we are all in this
  together. We all share the wish to be free from suffering. Probably an
  aggressor has forgotten this fact, hence his intention to harm us; but if we
  seek to harm him in return, only increased mutual suffering ensues. Regular
  recollection on the universality of suffering can protect us from falling into
  this vortex. Spending a short period of time each day considering how we are
  all seeking contentment can give rise to feelings of empathy and compassion.
  This is not an argument of which we will be convinced by reflection alone, but
  if we immerse ourselves in this contemplation we could find the benefit for
  ourselves.
\end{dhpRefl}

%% == 52+13 ==

\begin{dhpVerse}{123}
\label{dhp-123}
As one who is entrusted\\
with precious cargo\\
would remain vigilant and protective,\\
avoid evil as if it were poison.
\end{dhpVerse}

\begin{dhpRefl}
  We do have precious cargo, human consciousness. And we love life, hence the
  effort we make to protect it. The Buddha is saying we should watch over this
  good fortune which we have inherited by avoiding all evil actions. Great
  benefit can be discovered in this life if we are careful and cultivate wisdom.
  Similarly, great suffering can arise if we are heedless. Evil is a strong word
  and we might prefer not to use it, but we are naive not to contemplate it.
  When the heart is possessed by greed or hatred, evil actions may follow. Once
  they are performed there will be painful consequences. Nobody else can save us
  from heedlessness, not even the Buddha. Through kindness and wise reflection
  we contribute to the protection of all beings.
\end{dhpRefl}

%% == 52+14 ==

\begin{dhpVerse}{408}
\label{dhp-408}
Those who speak truth\\
and give gentle encouragement,\\
contending with no-one,\\
these do I call great beings.
\end{dhpVerse}

\begin{dhpRefl}
  There are times when we need to be assertive, as our body’s immune system is
  assertive when dealing with disease. But let’s not make \mbox{assertiveness} our only
  way of being. It can appear strong and impressive; it gets results; but it has
  limitations. There are times when gentleness is what is called for. Gentle
  speech which is true and encouraging also produces~results.
\end{dhpRefl}

%% == 52+15 ==

\begin{dhpVerse}{318}
\label{dhp-318}
Distorted views\\
which give rise to seeing right as wrong\\
and wrong as right\\
cause beings to disintegrate.
\end{dhpVerse}

\begin{dhpRefl}
  We sometimes need reminding that the causes of suffering, our own and that of
  the world, are complex. Often it is not what is happening in the outer world
  that leads to our struggles, but how we view things. Assuming the validity of
  views and opinions just because they are commonly held is not wise;
  convenient, perhaps, but that is not a good reason to invest in them. It is
  simplistic to collude with collective thinking without looking into the
  consequences. Having preferences is natural, but clinging to them and finding
  identity by holding to them leads to prejudice and disintegration, inwardly
  and outwardly.
\end{dhpRefl}

%% == 52+16 ==

\begin{dhpVerse}{291}
\label{dhp-291}
You will not succeed in your pursuit of happiness\\
if it is at the expense of others' well-being.\\
The snare of ill-will can still entangle you. 
\end{dhpVerse}

\begin{dhpRefl}
  Happiness is like food, it nourishes us. For happiness to be wholesome and
  genuinely sustaining, however, our efforts must come with empathy. Striving to
  be happy but lacking awareness of how we affect others is short-sighted. We
  may think we are generating causes for well-being, but in our hearts be
  harbouring jealousy or enmity.
\end{dhpRefl}

%% == 52+17 ==

\begin{dhpVerse}{151}
\label{dhp-151}
Passed down by the wise is the knowledge that\\
though what is externally impressive loses its splendour,\\
and though our bodies will decay,\\
the truth itself outlasts all degeneration.
\end{dhpVerse}

\begin{dhpRefl}
  Not only our bodies but all material objects are subject to the law of
  impermanence, as are social structures, institutions, relationships and
  organizations. Everything around us and within us is in a state of perpetual
  flux. For the sake of their balanced development it is necessary to protect
  children somewhat from this fact. For instance, we don't repeatedly remind
  them that their parents are mortal. But as we grow up, sooner or later we must
  admit that this is really how it is. Indeed, we need not just to admit this
  truth, but to embrace it, if we wish to accord wisely with changing
  conditions. The Buddha identified the law of impermanence as something beyond
  degeneration; something stable and secure, a Truth to which we can turn to
  find a frame of reference.
\end{dhpRefl}

%% == 52+18 ==

\begin{dhpVerse}{129}
\label{dhp-129}
Through having empathy for others,\\
one sees that all beings are afraid\\
of harm and death.\\
Knowing this, one does not kill\\
or cause to be killed.
\end{dhpVerse}

\begin{dhpRefl}
  Empathy is the essence of harmony. Throughout our lives we depend to varying
  degrees on others. If we forget that we all long for happiness and fear harm
  we risk being dominated by self-centred concerns, but we can learn to
  recognize that which we all share. Empathy supports insight into selflessness.
  Through empathy we see that others too, like us, hope not to be disappointed,
  and fear losing the things they hold dear. Even the wish to cause harm to
  another is a form of suffering we share with others. All those whose sense of
  identity comes from attaching to their body-mind are obliged to endure
  disharmony and the distorted thoughts and feelings which arise as a
  consequence. Letting go of attachment to this body-mind and recognizing our
  identity in understanding means disharmony simply won’t arise.
\end{dhpRefl}

%% == 52+19 ==

\begin{dhpVerse}{92}
\label{dhp-92}
Just like birds that leave no tracks in the air,\\
there are those whose minds do not cling\\
to temptations that are offered to them.\\
Their focus is the signless state of liberation,\\
which to others is indiscernible.
\end{dhpVerse}

\begin{dhpRefl}
  Can we do what we do so completely, so fully, that we leave no tracks behind?
  Probably not. Our habits of clinging mean we tend to do things partially. Our
  speech can be manipulative, leaving uncertainty behind. Our actions can be
  self-seeking, leaving feelings of incompleteness behind. And our thoughts can
  be all over the place, leaving us confused. This subtle image of birds flying
  through the sky, with no tracks left behind, inspires us to live without
  clinging. In making this effort we align ourselves with those great beings who
  have done what needs to be done, leaving nothing but Truth behind.
\end{dhpRefl}

%% == 52+20 ==

\begin{dhpVerse}{296}
\label{dhp-296}
Disciples of the Buddha\\
are fully awake,\\
dwelling both day and night\\
in contemplation of the Awakened One. 
\end{dhpVerse}

\begin{dhpRefl}
  We can admire our Teacher, the Buddha, without abandoning who and what we are
  right now. There are those who, when invited to dwell in contemplation of the
  spiritual master, betray themselves in their attempts to imitate another. The
  Buddha didn’t want us to ignore who we feel ourselves to be and pretend to be
  somebody else; rather, he encouraged an open-hearted, clear-minded receptivity
  of ‘this’ person, here and now, including all our limitations and obsessions.
  Taking on a new set of conditioned habits in an attempt to be free from
  suffering is abdication, not liberation. In practice we include all of
  ourselves in a vast field of awareness, free of discrimination and bias, and
  in so doing we offer all of ourselves in service to Dhamma.
\end{dhpRefl}

%% == 52+21 ==

\begin{dhpVerse}{285}
\label{dhp-285}
Remove the bonds of affection\\
as one might pluck an autumn flower.\\
Walk the Way that leads to liberation\\
explained by the Awakened One.
\end{dhpVerse}

\begin{dhpRefl}
  We will not free ourselves from attachments by holding to opinions about how
  'life' should be. ‘Life’ here refers to everything: self, others, material
  possessions. Even religious opinions lead to suffering if we cling to them in
  the wrong way. Freedom comes from recognizing how we hold onto things as we
  are holding onto them. Why do we resist the reality of change? Change is
  constant, yet we don’t see it. Walking the Awakened One’s Way to liberation
  means examining our relationship to all experience, agreeable and
  disagreeable. Every single moment of our life is an opportunity to learn how
  to let go, let be and understand.
\end{dhpRefl}